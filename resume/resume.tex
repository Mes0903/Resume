%-----------------------------------------------------------------------------------------------------------------------------------------------%
%	The MIT License (MIT)
%
%	Copyright (c) 2021 Philip Empl
%
%	Permission is hereby granted, free of charge, to any person obtaining a copy
%	of this software and associated documentation files (the "Software"), to deal
%	in the Software without restriction, including without limitation the rights
%	to use, copy, modify, merge, publish, distribute, sublicense, and/or sell
%	copies of the Software, and to permit persons to whom the Software is
%	furnished to do so, subject to the following conditions:
%	
%	THE SOFTWARE IS PROVIDED "AS IS", WITHOUT WARRANTY OF ANY KIND, EXPRESS OR
%	IMPLIED, INCLUDING BUT NOT LIMITED TO THE WARRANTIES OF MERCHANTABILITY,
%	FITNESS FOR A PARTICULAR PURPOSE AND NONINFRINGEMENT. IN NO EVENT SHALL THE
%	AUTHORS OR COPYRIGHT HOLDERS BE LIABLE FOR ANY CLAIM, DAMAGES OR OTHER
%	LIABILITY, WHETHER IN AN ACTION OF CONTRACT, TORT OR OTHERWISE, ARISING FROM,
%	OUT OF OR IN CONNECTION WITH THE SOFTWARE OR THE USE OR OTHER DEALINGS IN
%	THE SOFTWARE.
%	
%
%-----------------------------------------------------------------------------------------------------------------------------------------------%
% Original: https://www.overleaf.com/latex/templates/modern-latex-cv/qmdwjvcrcrph


%============================================================================%
%
%	DOCUMENT DEFINITION
%
%============================================================================%

\documentclass[10pt,A4,english]{article}	


%----------------------------------------------------------------------------------------
%	ENCODING
%----------------------------------------------------------------------------------------

% we use utf8 since we want to build from any machine
\usepackage[utf8]{inputenc}		
\usepackage[USenglish]{isodate}
\usepackage{fancyhdr}
\usepackage[numbers]{natbib}

%----------------------------------------------------------------------------------------
%	LOGIC
%----------------------------------------------------------------------------------------

% provides \isempty test
\usepackage{xstring, xifthen}
\usepackage{enumitem}
\usepackage[english]{babel}
\usepackage{blindtext}
\usepackage{pdfpages}
\usepackage{changepage}
\usepackage{calc}
%----------------------------------------------------------------------------------------
%	FONT BASICS
%----------------------------------------------------------------------------------------

% some tex-live fonts - choose your own

%\usepackage[defaultsans]{droidsans}
%\usepackage[default]{comfortaa}
%\usepackage{cmbright}
\usepackage[default]{raleway}
%\usepackage{fetamont}
%\usepackage[default]{gillius}
%\usepackage[light,math]{iwona}
%\usepackage[thin]{roboto} 

% set font default
\renewcommand*\familydefault{\sfdefault} 	
\usepackage[T1]{fontenc}

% more font size definitions
\usepackage{moresize}

\usepackage{xeCJK}
\setCJKmonofont{微軟正黑體}

%----------------------------------------------------------------------------------------
%	FONT AWESOME ICONS
%---------------------------------------------------------------------------------------- 

% include the fontawesome icon set
\usepackage{fontawesome}

% use to vertically center content
% credits to: http://tex.stackexchange.com/questions/7219/how-to-vertically-center-two-images-next-to-each-other
\newcommand{\vcenteredinclude}[1]{\begingroup
	\setbox0=\hbox{\includegraphics{#1}}%
	\parbox{\wd0}{\box0}\endgroup}
\newcommand{\tab}[1]{\hspace{.2\textwidth}\rlap{#1}}
% use to vertically center content
% credits to: http://tex.stackexchange.com/questions/7219/how-to-vertically-center-two-images-next-to-each-other
\newcommand*{\vcenteredhbox}[1]{\begingroup
	\setbox0=\hbox{#1}\parbox{\wd0}{\box0}\endgroup}

% icon shortcut
\newcommand{\icon}[3] { 							
	\makebox(#2, #2){\textcolor{maincol}{\csname fa#1\endcsname}}
}	


% icon with text shortcut
\newcommand{\icontext}[4]{ 						
	\vcenteredhbox{\icon{#1}{#2}{#3}}  \hspace{2pt}  \parbox{0.9\mpwidth}{\textcolor{#4}{#3}}
}

% icon with website url
\newcommand{\iconhref}[5]{ 						
	\vcenteredhbox{\icon{#1}{#2}{#5}}  \hspace{2pt} \href{#4}{\textcolor{#5}{#3}}
}

% icon with email link
\newcommand{\iconemail}[5]{ 						
	\vcenteredhbox{\icon{#1}{#2}{#5}}  \hspace{2pt} \href{mailto:#4}{\textcolor{#5}{#3}}
}

%----------------------------------------------------------------------------------------
%	PAGE LAYOUT  DEFINITIONS
%----------------------------------------------------------------------------------------

% page outer frames (debug-only)
% \usepackage{showframe}		

% we use paracol to display breakable two columns
\usepackage{paracol}
\usepackage{tikzpagenodes}
\usetikzlibrary{calc}
\usepackage{lmodern}
\usepackage{multicol}
\usepackage{lipsum}
\usepackage{atbegshi}
% define page styles using geometry
\usepackage[a4paper]{geometry}

% remove all possible margins
\geometry{top=1cm, bottom=1cm, left=1cm, right=1cm}

\usepackage{fancyhdr}
\pagestyle{empty}

% space between header and content
% \setlength{\headheight}{0pt}

% indentation is zero
\setlength{\parindent}{0mm}

%----------------------------------------------------------------------------------------
%	TABLE /ARRAY DEFINITIONS
%---------------------------------------------------------------------------------------- 

% extended aligning of tabular cells
\usepackage{array}

% custom column right-align with fixed width
% use like p{size} but via x{size}
\newcolumntype{x}[1]{%
	>{\raggedleft\hspace{0pt}}p{#1}}%


%----------------------------------------------------------------------------------------
%	GRAPHICS DEFINITIONS
%---------------------------------------------------------------------------------------- 

%for header image
\usepackage{graphicx}

% use this for floating figures
% \usepackage{wrapfig}
% \usepackage{float}
% \floatstyle{boxed} 
% \restylefloat{figure}

%for drawing graphics		
\usepackage{tikz}			
\usepackage{ragged2e}	
\usetikzlibrary{shapes, backgrounds,mindmap, trees}

%----------------------------------------------------------------------------------------
%	Color DEFINITIONS
%---------------------------------------------------------------------------------------- 
\usepackage{transparent}
\usepackage{color}

% primary color
\definecolor{maincol}{RGB}{ 64,64,64}

% accent color, secondary
% \definecolor{accentcol}{RGB}{ 250, 150, 10 }

% dark color
\definecolor{darkcol}{RGB}{ 70, 70, 70 }

% light color
\definecolor{lightcol}{RGB}{245,245,245}

\definecolor{accentcol}{RGB}{59,77,97}



% Package for links, must be the last package used
\usepackage[hidelinks]{hyperref}

% returns minipage width minus two times \fboxsep
% to keep padding included in width calculations
% can also be used for other boxes / environments
\newcommand{\mpwidth}{\linewidth-\fboxsep-\fboxsep}



%============================================================================%
%
%	CV COMMANDS
%
%============================================================================%

%----------------------------------------------------------------------------------------
%	 CV LIST
%----------------------------------------------------------------------------------------

% renders a standard latex list but abstracts away the environment definition (begin/end)
\newcommand{\cvlist}[1] {
	\begin{itemize}{#1}\end{itemize}
}

%----------------------------------------------------------------------------------------
%	 CV TEXT
%----------------------------------------------------------------------------------------

% base class to wrap any text based stuff here. Renders like a paragraph.
% Allows complex commands to be passed, too.
% param 1: *any
\newcommand{\cvtext}[1] {
	\begin{tabular*}{1\mpwidth}{p{0.98\mpwidth}}
		\parbox{1\mpwidth}{#1}
	\end{tabular*}
}
\newcommand{\cvtextsmall}[1] {
	\begin{tabular*}{0.8\mpwidth}{p{0.8\mpwidth}}
		\parbox{0.8\mpwidth}{#1}
	\end{tabular*}
}
%----------------------------------------------------------------------------------------
%	CV SECTION
%----------------------------------------------------------------------------------------

% Renders a a CV section headline with a nice underline in main color.
% param 1: section title
\newcommand{\cvsection}[1] {
	\vspace{14pt}
	\cvtext{
		\textbf{\LARGE{\textcolor{darkcol}{#1}}}\\[-4pt]
		\textcolor{accentcol}{ \rule{0.2\textwidth}{1.5pt} } \\
	}
}

\newcommand{\cvsectionsmall}[1] {
	\vspace{14pt}
	\cvtext{
		\textbf{\Large{\textcolor{darkcol}{#1}}}\\[-4pt]
		\textcolor{accentcol}{ \rule{0.2\textwidth}{1.5pt} } \\
	}
}

\newcommand{\cvheadline}[1] {
	\vspace{16pt}
	\cvtext{
		\textbf{\Huge{\textcolor{accentcol}{#1}}}\\[-4pt]
		
	}
}

\newcommand{\cvsubheadline}[1] {
	\vspace{16pt}
	\cvtext{
		\textbf{\huge{\textcolor{darkcol}{#1}}}\\[-4pt]
		
	}
}
%----------------------------------------------------------------------------------------
%	META SKILL
%----------------------------------------------------------------------------------------

% Renders a progress-bar to indicate a certain skill in percent.
% param 1: name of the skill / tech / etc.
% param 2: level (for example in years)
% param 3: percent, values range from 0 to 1
\newcommand{\cvskill}[3] {
	\begin{tabular*}{1\mpwidth}{p{0.8\mpwidth}  r}
		\textcolor{black}{\textbf{#1}} & \hspace*{-0.5cm} \textcolor{maincol}{#2}\\
	\end{tabular*}%
	
	\hspace{4pt}
	\begin{tikzpicture}[scale=1,rounded corners=2pt,very thin]
		\fill [lightcol] (0,0) rectangle (1\mpwidth, 0.15);
		\fill [accentcol] (0,0) rectangle (#3\mpwidth, 0.15);
	\end{tikzpicture}%
}


%----------------------------------------------------------------------------------------
%	 CV EVENT
%----------------------------------------------------------------------------------------

% Renders a table and a paragraph (cvtext) wrapped in a parbox (to ensure minimum content
% is glued together when a pagebreak appears).
% Additional Information can be passed in text or list form (or other environments).
% the work you did
% param 1: time-frame i.e. Sep 14 - Jan 15 etc.
% param 2:	 event name (job position etc.)
% param 3: Customer, Employer, Industry
% param 4: Short description
% param 5: work done (optional)
% param 6: technologies include (optional)
% param 7: achievements (optional)
\newcommand{\cvevent}[7] {
	
	% we wrap this part in a parbox, so title and description are not separated on a pagebreak
	% if you need more control on page breaks, remove the parbox
	\parbox{\mpwidth}{
		\begin{tabular*}{1\mpwidth}{p{0.66\mpwidth}  r}
			\fontsize{14}{12}\selectfont\textcolor{black}{\textbf{#2}} & \colorbox{accentcol}{\makebox[0.32\mpwidth]{\textcolor{white}{\textbf{#1}}}} \\
			\textcolor{black}{#3}
		\end{tabular*}
		
		\ifthenelse{\isempty{#4}}{}{
			#4
		}
	}
}


%----------------------------------------------------------------------------------------
%	 CV META EVENT
%----------------------------------------------------------------------------------------

% Renders a CV event on the sidebar
% param 1: title
% param 2: subtitle (optional)
% param 3: customer, employer, etc,. (optional)
% param 4: info text (optional)
\newcommand{\cvmetaevent}[2] {
	\textcolor{maincol} { \cvtext{\textbf{\begin{flushleft}#1\end{flushleft}}}}
	
	\ifthenelse{\isempty{#2}}{}{
		\textcolor{black} {\cvtext{\textbf{#2}} }
	}
}

%---------------------------------------------------------------------------------------
%	QR CODE
%----------------------------------------------------------------------------------------

% Renders a qrcode image (centered, relative to the parentwidth)
% param 1: percent width, from 0 to 1
% \newcommand{\cvqrcode}[1] {
% 	\begin{center}
% 		\includegraphics[width={#1}\mpwidth]{qrcode}
% 	\end{center}
% }


% HEADER AND FOOOTER 
%====================================
\newcommand\Header[1]{%
	\begin{tikzpicture}[remember picture,overlay]
		\fill[accentcol]
		(current page.north west) -- (current page.north east) --
		([yshift=50pt]current page.north east|-current page text area.north east) --
		([yshift=50pt,xshift=-3cm]current page.north|-current page text area.north) --
		([yshift=10pt,xshift=-5cm]current page.north|-current page text area.north) --
		([yshift=10pt]current page.north west|-current page text area.north west) -- cycle;
		\node[font=\sffamily\bfseries\color{white},anchor=west,
		xshift=0.7cm,yshift=-0.32cm] at (current page.north west)
		{\fontsize{12}{12}\selectfont {#1}};
	\end{tikzpicture}%
}

\newcommand\Footer[1]{%
	\begin{tikzpicture}[remember picture,overlay]
		\fill[lightcol]
		(current page.south east) -- (current page.south west) --
		([yshift=-80pt]current page.south east|-current page text area.south east) --
		([yshift=-80pt,xshift=-6cm]current page.south|-current page text area.south) --
		([xshift=-2.5cm,yshift=-10pt]current page.south|-current page text area.south) --	
		([yshift=-10pt]current page.south east|-current page text area.south east) -- cycle;
		\node[yshift=0.32cm,xshift=9cm] at (current page.south) {\fontsize{10}{10}\selectfont \textbf{\thepage}};
	\end{tikzpicture}%
}


%=====================================
%============================================================================%
%
%
%
%	DOCUMENT CONTENT
%
%
%
%============================================================================%
\begin{document}
	
	\columnratio{0.31}
	\setlength{\columnsep}{2.2em}
	\setlength{\columnseprule}{4pt}
	\colseprulecolor{white}
	
	
	% LEBENSLAUF HIERE
	\AtBeginShipoutFirst{\Header{CV}\Footer{1}}
	\AtBeginShipout{\AtBeginShipoutAddToBox{\Header{CV}\Footer{2}}}
	
	\newpage
	
	\colseprulecolor{lightcol}
	\columnratio{0.31}
	\setlength{\columnsep}{2.2em}
	\setlength{\columnseprule}{4pt}
	\begin{paracol}{2}
		
		
		\begin{leftcolumn}
			%---------------------------------------------------------------------------------------
			%	META IMAGE
			%----------------------------------------------------------------------------------------
			%\includegraphics[width=\linewidth]{resources/image.jpg}	%trimming relative to image size
			
			%---------------------------------------------------------------------------------------
			%	META SKILLS
			%----------------------------------------------------------------------------------------
			\fcolorbox{white}{white}{\begin{minipage}[c][1.5cm][c]{1\mpwidth}
					\LARGE{\textbf{\textcolor{maincol}{鄭詠澤}}} \\[2pt]
					\normalsize{ \textcolor{maincol} {MIS of NCU Mathematics} }
			\end{minipage}} \
		
			\cvsection{Education}

      \vspace*{-14pt}
			\cvmetaevent
			{2020 - 2024}
			{Applied Mathematics\newline National Central University}


			\cvsection{Skills}
			
      \cvskill{作業系統 \& 嵌入式系統} {} {1} \\[-2pt]

			\cvskill{程式語言 \fontsize{7}{1}\selectfont(C/C++ \& 形式語言)} {} {1} \\[-2pt]
			
			\cvskill{電腦圖學} {} {0.8} \\[-2pt]
			
      \cvskill{計算機網路與系統管理} {} {0.7} \\[-2pt]

      \cvskill{資訊安全 \fontsize{7}{1}\selectfont(Web \& Reverse)} {} {0.4} \\[-2pt]

			Language skills\\
			
			\cvskill{Chinese} {Native} {1} \\[-2pt]
			
			\cvskill{English} {} {0.8} \\[-2pt]
			
			\cvskill{Japanese} {} {0.5} \\[-2pt]

      \cvskill{Cantonese} {} {0.3}
			
      
			\cvsection{Additional Skills}

			\icontext{CaretRight}{12}{RISC-V, ARM, x86, Lua, verilog, MATLAB, Python, Rust, \LaTeX}{black}\\[4pt]
      \icontext{CaretRight}{12}{SQL, Laravel, TypeScript, CSS}{black}\\[4pt]
			\icontext{CaretRight}{12}{OpenGL, GLFW, ImGui, CMake, Qt, Gem5, Boost, Cisco Tos, Git}{black}\\[4pt]
      \icontext{CaretRight}{12}{Compiler, Non-volatile memory, Wireless \& Cellular Network, Machine Learning, CAD for VLSI}{black}
      \icontext{CaretRight}{12}{Docker, Promox, Kubernetes}{black}

      \vspace*{-14pt}
			\cvsection{Contact}
			
			%\icontext{MapMarker}{16}{Street name XX\\D-XXXXX Lorem}{black}\\[6pt]
			%\iconhref{Home}{16}{XXX.XXX-XXXX.XX}{http://XXX-XXX.XXX.XX}{black}\\[6pt]
			\icontext{MobilePhone}{16}{+886 0935528966}{black}\\[6pt]
			\iconemail{Envelope}{16}{mes900903@gmail.com}{mes900903@gmail.com}{black}\\[6pt]
			\iconhref{Github}{16}{github.com/Mes0903}{https://github.com/Mes0903}{black}\\[6pt]
			\iconhref{Globe}{16}{mes0903.github.io}{https://mes0903.github.io/index.html}{black}\\[6pt]
			%\iconhref{Xing}{16}{xing.com/user\_name}{https://www.xing.com/profile/User_Name}{black}\\
      

      \cvsection{Courses}
      
      嵌入式系統開發相關\\

      \cvskill{電腦攻擊與防禦} {100} {1} \\[-2pt]

      \cvskill{Linux 作業系統} {97} {0.97} \\[-2pt]

      \cvskill{新興記憶體儲存系統元件} {92} {0.92} \\[-2pt]

      \cvskill{作業系統} {85} {0.85} \\[-2pt]

      \cvskill{數位邏輯實驗} {81} {0.81}\\[-2pt]
      
      \cvskill{計算機結構} {77} {0.77} \\[-2pt]  

      程式語言相關\\
      
      \cvskill{程式語言} {99} {0.99} \\[-2pt]
      
      \cvskill{程式語言及其應用} {97} {0.97} \\[-2pt]
      
      \cvskill{基礎程式設計} {85} {0.85} \\[-2pt]
      
      \cvskill{資料結構} {84} {0.84} \\[-2pt]
      
      \cvskill{計算機概論} {80} {0.8} \\[-2pt]

      數學應用相關\\
      
      \cvskill{資料科學導論} {92} {0.92} \\[-2pt]
      
      \cvskill{科學計算導論} {91} {0.91} \\[-2pt]
      
      \cvskill{3D 計算機圖學} {85} {0.85} \\[-2pt]
      
      專題\\
      
      \cvskill{資料科學專題} {88} {0.88} \\[-2pt]
      
      \cvskill{專題實驗 II \fontsize{7}{1}\selectfont (作業系統實作)} {98} {0.98} \\[-2pt]
      
      \cvskill{專題實驗 III \fontsize{7}{1}\selectfont (作業系統實作)} {99} {0.99} \\[-2pt]

			\vspace*{\fill}
      Compiled \today\xspace with \LaTeX

		\end{leftcolumn}
		\begin{rightcolumn}
			%---------------------------------------------------------------------------------------
			%	TITLE  HEADER
			%----------------------------------------------------------------------------------------
			
			
			%---------------------------------------------------------------------------------------
			%	PROFILE
			%----------------------------------------------------------------------------------------
			\cvsection{Profile}
			
			\textcolor{black}{\cvtext{
				現為中央大學數學系大四學生,目前擔任數學系網管。 大一入學至今已完成一項產學合作、一項教學實踐計畫及一項高教深耕計畫。 目前正協助團隊執行五項高教深耕計畫。 \\[4pt]
        擅長的領域為形式化語言、嵌入式系統與電腦圖學。 夢想為自己從硬體設計到作業系統開發來製作一台遊戲主機。 目前致力於開發自己的作業系統與貢獻 Linux kernel 及 FreeBSD。
			}}
			
			
			%---------------------------------------------------------------------------------------
			%	WORK EXPERIENCE
			%----------------------------------------------------------------------------------------
			
			\vspace{4pt}
			\cvsection{Work experience}
			\vspace{4pt}
			
			\cvevent
			{07/2023 - CURRENT}
			{子由數學小學堂維護 \& 開發}
			{數學系計算機中心}
			{\begin{itemize}
        \item 每學期固定修復小學堂資安漏洞,使其通過教育部指定弱點掃描認證
        \item 配合產學合作與高教深耕計畫開發新功能
        \item 領導團隊執行 2024 年的重構計畫,並負責重構其中的題目出題引擎端
        \end{itemize}}
			\vspace*{10pt}\vfill\null

			\cvevent
			{07/2022 - CURRENT}
			{網管}
			{數學系計算機中心}
			{\begin{itemize}
        \item 維護系上網頁及伺服器
        \item 培訓計算機中心工讀生及協助其執行系上計畫
        \item 開發及架設系上服務需求,如 gpu server、電腦教室 NAT \& DHCP server 與 NAS server 等
      \end{itemize}}
      \vspace*{10pt}\vfill\null
			
      \cvevent
			{09/2021 - 02/2022}
			{工讀生}
			{機器人實驗室}
			{\begin{itemize}
        \item 維護系上教學用機器人
        \item 協助「機器人專題」與「資料科學導論/專題」修課生解決上課及作業問題
        \item 擔任全校開放性教學 ROS Tutorial 之主講者
        \item 開發教學實踐計畫「系狗計畫:人工智慧教學平台」的教學用機器狗,負責其中的 IMU 與 3D 零件列印部分
      \end{itemize}}
      \vfill\null

			\cvevent
			{09/2020 - 07/2022}
			{工讀生}
			{數學系計算機中心}
			{\begin{itemize}
          \item 維護系上教學設備,如系辦及教室電腦、教室投影機與廣播設備等
          \item 執行系上高教深耕計畫
      \end{itemize}}
      \vspace*{10pt}\vfill\null
			
			
      \vspace*{-10pt}
			\cvsection{Graduation projects}

      \cvevent
      {02/2023 - 02/2024}
      {作業系統實作 \href{https://github.com/Mes0903/MesRVOS.git}{\fontsize{9}{1}\selectfont[\textcolor{blue}{github}]} \href{https://mes0903.github.io/2023/06/17/OS/OSDI/risc-v_osdi/}{\fontsize{9}{1}\selectfont[\textcolor{blue}{note}]}}
      {資工系專題\vspace*{2pt}}
      {\cvtext{於 QEMU 上利用 RISC-V 與 C/C++ 實作一個簡單的作業系統,擁有簡單的 bootloader、page、interrupt、context switch、preemtive multitasking 與 systemcall}}
      \vspace*{10pt}\vfill\null

      \cvevent
      {02/2023 - 02/2024}
      {惡意軟體偵測與分類 \href{https://github.com/Mes0903/mes_malware_detect.git}{\fontsize{9}{1}\selectfont[\textcolor{blue}{github}]} \href{https://youtu.be/gU_u7FEmX64?t=9897}{\fontsize{9}{1}\selectfont[\textcolor{blue}{video}]}}
      {數學系專題\vspace*{2pt}}
      {\cvtext{以 windows PE format 與執行檔的 opcode 分布作為特徵,在僅使用線代函式庫的情況下實作了 Random Forest 與 Adaboost 算法,並對 windows 上的惡意軟體進行偵測與分類}}
      \vspace*{10pt}\vfill\null
      
      \newpage

			\cvsection{Projects}
      \newline
			\cvtext {
				一些比較重要的 side projects 與課堂上的專案
			}
			\vspace{4pt}

      \vspace*{10pt}
      \cvevent
      {2020 - CURRENT}
      {CPP-Miner \href{https://github.com/Mes0903/Cpp-Miner.git}{\fontsize{9}{1}\selectfont[\textcolor{blue}{github}]}}
      {side project\vspace*{2pt}}
      {\cvtext{古典 Cpp 基礎與 Modern Cpp 教學\& Cpp Spec/Committee Paper 的整理。 目前於 github 上有 137 顆星星}}
      \vspace*{10pt}\vfill\null

      \vspace*{10pt}
      \cvevent
      {2021}
      {Mase \href{https://github.com/Mes0903/Mase.git}{\fontsize{9}{1}\selectfont[\textcolor{blue}{github}]} \href{https://www.youtube.com/watch?v=hkr8zXTF1jI}{\fontsize{9}{1}\selectfont[\textcolor{blue}{video}]}}
      {side project\vspace*{2pt}}
      {\cvtext{採用 MVC 架構,使用 Qt 來實現演算法視覺化。 實作三種生成樹算法用以自動生成迷宮,實作五種演算法,如 UCS 與 A* 等古典 AI 演算法解迷宮}}
      \vspace*{10pt}\vfill\null
      
      \vspace*{10pt}
      \cvevent
      {2021 - 2022}
      {ROS 機器人導航 \href{https://youtu.be/ogBy5Eq6Nds?t=52}{\fontsize{9}{1}\selectfont[\textcolor{blue}{video}]}}
      {機器人專題\vspace*{2pt}}
      {\cvtext{整合 ROS 與 Github 上的開源專案製作了一個機器人導航系統,使用的是輪型機器人,定位使用 AMCL 演算法,導航使用 Dijkstra 與 A* 演算法}}
      \vspace*{10pt}\vfill\null

      \vspace*{10pt}
      \cvevent
      {2022}
      {Chat Server/Client \href{https://github.com/Mes0903/MesTCP}{\fontsize{9}{1}\selectfont[\textcolor{blue}{github}]} \href{https://youtu.be/ThSNljCJOxY?t=3282}{\fontsize{9}{1}\selectfont[\textcolor{blue}{video}]}}
      {程式語言及其應用專案\vspace*{2pt}}
      {\cvtext{使用 C++ 配合 boost asio 實作了一個廣播型的聊天室,內部使用 asio socket 對接並設計好完整的 socket 生命週期與資料包以完成使用者間的通訊}}
      \vspace*{10pt}\vfill\null
			
      \vspace*{10pt}
      \cvevent
      {2022}
      {Simple compiler of iscas85 \href{https://github.com/Mes0903/NCU_EDA}{\fontsize{9}{1}\selectfont[\textcolor{blue}{github}]}}
      {CAD for VLSI Design 專案\vspace*{2pt}}
      {\cvtext{使用 C++ 針對 iscas85 的 module 寫了一個簡單的 compiler,實作了簡單的中間層語言,並對電路進行優化,最終將其轉換為 verilog 輸出}}
      \vspace*{10pt}\vfill\null

      \vspace*{10pt}
      \cvevent
      {2023}
      {球與箱子的實時分類 \href{https://github.com/Mes0903/mes_detect_ball.git}{\fontsize{9}{1}\selectfont[\textcolor{blue}{github}]} \href{https://www.youtube.com/watch?v=rctCARWYpT4}{\fontsize{9}{1}\selectfont[\textcolor{blue}{video}]}}
      {資料科學導論專案\vspace*{2pt}}
      {\cvtext{以球體的徵圓度、線性度與 hit box 等製作了十個特鄭,在僅使用線代函式庫的情況下實作了 Adaboost 算法,最後達到 97\% 的 F1-score}}
      \vspace*{10pt}\vfill\null

      \vspace*{10pt}
      \cvevent
      {2023}
      {圖學演算法實作 \href{https://github.com/Mes0903/NCUComputerGraphic.git}{\fontsize{9}{1}\selectfont[\textcolor{blue}{github}]}}
      {3D 計算機圖學專案\vspace*{2pt}}
      {\cvtext{實作了線條、圓圈及橢圓曲線等圖學算法,最後組成了一個簡單的繪畫軟體。 另外還實作了 MVP 轉換,搭配按鍵輸入組成了一個簡單的 3D 顯示器}}
      \vspace*{10pt}\vfill\null
    
      \vspace*{10pt}
      \cvevent
      {2023}
      {User Space FileSystem \href{https://github.com/Mes0903/FuseCpp}{\fontsize{9}{1}\selectfont[\textcolor{blue}{github}]}}
      {新興記憶儲存系統元件設計專案\vspace*{2pt}}
      {\cvtext{使用 C++ 搭配 libfuse 實作了一個簡單的 user-space filesystem,並支援簡單的 bash 指令,如 cd, ls, mkdir, rmdir, touch, rm, echo, cat 等}}
      \vspace*{10pt}\vfill\null


      \cvsection{Projects of NCU Mathematics}
      \newline
			\cvtext {
				一些比較重要的系上產學合作、教學實踐計畫與高教深耕計畫
			}

      \vspace*{10pt}
      \cvevent
      {Developer}
      {系狗計畫:人工智慧教學平台 \href{https://www.youtube.com/watch?v=dDUmEt6-t3w}{\fontsize{9}{1}\selectfont[\textcolor{blue}{video}]}}
      {教學實踐計畫\vspace*{2pt}}
      {}
      \vspace*{10pt}\vfill\null

      \cvevent
      {Developer}
      {Hyread 產學合作}
      {產學合作\vspace*{2pt}}
      {}
      \vspace*{10pt}\vfill\null

      \cvevent
      {Developer}
      {資料科學修課用 Label Tool \href{https://www.youtube.com/watch?v=m87yXTsJ6vg}{\fontsize{9}{1}\selectfont[\textcolor{blue}{video}]}}
      {高教深耕計畫\vspace*{2pt}}
      {}
      \vspace*{10pt}\vfill\null

      \cvevent
      {Leader \& Developer}
      {子由數學小學堂 AI 推薦系統}
      {高教深耕計畫\vspace*{2pt}}
      {}
      \vspace*{10pt}\vfill\null

      \cvevent
      {Leader}
      {子由數學小學堂 app}
      {高教深耕計畫\vspace*{2pt}}
      {}
      \vspace*{10pt}\vfill\null

			%\today     \hspace{1cm}   \hrulefill
			%\hspace*{30mm}\phantom{Lorem, \today }Martin Chang
			
		\end{rightcolumn}
	\end{paracol}
	
	
\end{document}